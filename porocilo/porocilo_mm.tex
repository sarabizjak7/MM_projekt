\documentclass[a4paper,12pt]{article}

\usepackage[slovene]{babel}
\usepackage[utf8]{inputenc}
\usepackage[T1]{fontenc}
\usepackage{lmodern}
\usepackage{amsmath}
\usepackage{amssymb}
\usepackage{amsthm}
\usepackage{amsfonts}
\usepackage{mathtools}
\usepackage{enumitem}



\begin{document}


%%% naslovna stran

\begin{titlepage}
    \centering
    \vfill
    \vfill
    \textbf{\Huge{Poročilo projekta}}
    \vfill
    \textbf{\LARGE{Matematično modeliranje}}
    \vfill\vfill\vfill\vfill\vfill
    \textsc{\Large{Sara Bizjak}}
    \vfill\vfill
    \textsc{\large{Univerza v Ljubljani}}
    
    \textsc{\large{Fakulteta za matematiko in fiziko}}
    
    \textsc{\large{Oddelek za matematiko}}
    \vfill\vfill
        
    \large{avgust 2019}
    
    \end{titlepage}

\newpage

\tableofcontents

\newpage

\section{\textsc{\large{PREDSTAVITEV PROBLEMA}}}

Rešujemo problem otroka, ki se sprehaja po ravnem igrišču na mivki, za seboj pa vleče na 
vrvico privezano igračo tako, da je vrvica vseskozi napeta. Otrokovo gibanje opišemo s 
parametrično krivuljo. Program izračuna sled gibanja igrače po mivki in izriše animacijo.
\\

\section{\textsc{\large{MATEMATIČNO OZADJE}}}

Rešujemo primer naloge, kjer gibanje igrače določimo z rešitvijo diferencialne enačbe. \\
Ker lahko za rešitev diferencialne enačbe v Matlabu uporabimo že vgrajeno funkcijo {\texttt{ode45}}, bomo reševanje diferencialnih enačb izpustili. 
Ponovimo samo nekaj osnovnih pojmov.
\\
Ker imamo podatek o hitrosti, rešujemo DE prvega reda.
\\
\\
Enačbi, v kateri nastopa neznana funkcija in njen odvod, pravimo \textbf{diferencialna enačba prvega reda}. 
Najsplošnejša oblika diferencialne enačbe prvega reda je
\begin{displaymath}
F(x, y, dy) = 0,
\end{displaymath}
kjer je \textit{F} dana funkcija treh spremenljivk, $y = y(x)$ pa je neznana funkcija.
Smiselno je zahtevati, da je definicijsko območje neznane funkcije odprt interval,
sicer imamo težave z računanjem njenih odvodov.
Pogosto lahko iz $F(x, y, dy) = 0$ izrazimo $dy$ kot funkcijo $x$ in $y$. V tem primeru pravimo,
da smo diferencialno enačbo prevedli na \textbf{standardno obliko}.
\\
\\
Vgrajena Matlabova funkcija za reševanje diferencialnih enačb izgleda takole:
\begin{displaymath}
[X, Y] = ode45(odefun, [x0, b], y0).
\end{displaymath}

\newpage
\section{\textsc{\large{REŠITEV PROBLEMA}}}

Poznamo parametrizacijo krivulje, po kateri se premika otrok. Ker je vrvica med otrokom in 
igračo vedno napeta, vemo, da imata v vsakem trenutku isto hitrost, oz. sta vedno enako 
oddaljena. Igrača se vedno giba v smeri otroka. 
Torej poznamo smer in velikost hitrosti igrače, njene pozicije oz. krivuljo gibanja pa 
dobimo kot rešitev diferencialne enačbe.
\\
\\
Z $x(t)$ in $y(y)$ je označena parametrizacija otroka, z $x_i(t)$ in $y_i(t)$ pa igrače. Odvodi so označeni kot $dx(t), \ dy(t), \ dx_i(t)$ in $dy_i(t)$. \\
Poznamo parametrizacijo otroka, torej $x(t)$ in $y(t)$. Vemo, da vektor hitrosti igrače kaže v smeri proti otroku (po vrvici).
Torej velja

\begin{align*}
    dx_i(t) = x(t) - x_i(t) \\
    dy_i(t) = y(t) - y_i(t).
\end{align*}
Vemo tudi, da sta hitrost otroka in hitrost igrače po velikosti enaki, torej velja
\begin{align*}
    \|(dx_i(t), dy_i(t))\| = \|(dx(t), dy(t))\|.
\end{align*}
Ker poznamo smer igrače in velikost hitrosti, lahko ugotovimo njeno parametrizacijo oziroma gibanje.
Vektor smeri normiramo in ga pomnožimo z normo hitrosti otroka, saj se velikosti hitrosti ujemata.
\\
\begin{align}
     \Big(
    \begin{bmatrix} 
        x(t) \\
        y(t)
    \end{bmatrix}
    -
    \begin{bmatrix} 
        x_i(t) \\
        y_i(t) 
    \end{bmatrix}
    \Big)
    \ \cdot \
    \frac{
    \Big \|
    \begin{bmatrix} 
        dx(t) \\
        dy(t)
    \end{bmatrix}
    \Big \|
    }
    {
    \Big \|
    \begin{bmatrix} 
        x(t) \\
        y(t)
    \end{bmatrix}
    -
    \begin{bmatrix} 
        x_i(t) \\
        y_i(t) 
    \end{bmatrix}
    \Big \|
    }
\end{align}
\\
\\
Da ohranjamo razdaljo oz. napeto vrvico med otrokom in igračo, račun (1) pomnožimo še 
s kotom med hitrostjo otroka in daljico, ki povezuje igračo in otroka. 
Velikost kota dobimo s skalarnim produktom, in sicer

\begin{align}
    \varphi = 
    \frac{
    \begin{bmatrix} 
        dx(t), \ %
        dy(t)
    \end{bmatrix}
    \cdot
    \begin{bmatrix} 
        x(t) - x_i(t) \\
        y(t) - y_i(t)
    \end{bmatrix}
    }
    {
    \Big \|
    \begin{bmatrix} 
        dx(t) \\
        dy(t)
    \end{bmatrix}
    \Big \| 
    \cdot
    \Big \|
    \begin{bmatrix} 
        x(t) - x_i(t) \\
        y(t) - y_i(t)
    \end{bmatrix}
    \Big \| 
    }
    .
\end{align}
\\
\\
Če pomnožimo (1) in (2), dobimo sistem \\
\begin{align}
    \begin{bmatrix} 
        dx_i(t) \\
        dy_i(t) 
    \end{bmatrix}
    =
    \frac{
    \begin{bmatrix} 
        x(t) - x_i(t) \\
        y(t) - y_i(t)
    \end{bmatrix}
    \cdot
    \Big(
    \begin{bmatrix} 
        dx(t), \ %
        dy(t)
    \end{bmatrix}
    \cdot
    \begin{bmatrix} 
        x(t) - x_i(t) \\
        y(t) - y_i(t)
    \end{bmatrix}
    \Big)
    }
    {
    \Big \|
    \begin{bmatrix} 
        x(t) - x_i(t) \\
        y(t) - y_i(t)
    \end{bmatrix}
    \Big \| ^ {2}
    }
    .
\end{align}
\\
\\
Sistem (3) sovpada s funkcijo, ki jo vstavimo v Matlabovo vgrajeno funkcijo za reševanje diferencialnih enačb 
{\texttt{ode45}}.
\\
\\
\begin{align*}
    odefun = @(t,\ P) 
    \
    \frac{
        \begin{bmatrix} 
            x(t) - x_i(t) \\
            y(t) - y_i(t)
        \end{bmatrix}
        \cdot
        \Big(
        \begin{bmatrix} 
            dx(t), \ %
            dy(t)
        \end{bmatrix}
        \cdot
        \begin{bmatrix} 
            x(t) - x_i(t) \\
            y(t) - y_i(t)
        \end{bmatrix}
        \Big)
        }
        {
        \Big \|
        \begin{bmatrix} 
            x(t) - x_i(t) \\
            y(t) - y_i(t)
        \end{bmatrix}
        \Big \| ^ {2}
        },
\end{align*}

\begin{align*}
    kjer \ P = 
        \begin{bmatrix} 
            x_i(t) \\
            y_i(t)
        \end{bmatrix}
    .
    \\
\end{align*}
Pri reševanju je predpostavljeno, da je vrvica dolga toliko, kot je začetna oddaljenost igrače in otroka. 
Igrača svoje gibanje vedno začne v koordinatnem izhodišču, torej v točki (0, 0), kar je tudi nastavljen začetni pogoj pri reševanju diferencialne enačbe. 


\subsection{\textsc{Matlabove datoteke}}

Rešitev diferencialne enačbe izračunamo s funkcijo v datoteki {\texttt{igraca.m}}. Datoteki {\texttt{risi\_otrok.m}} in {\texttt{risi\_igraca.m}} izriseta krivuljo, 
po kateri se gibata otrok oziroma igrača. Gibanje izrišemo s pomočjo {\texttt{animacija.m}}. 
Vse skupaj poženemo z datoteko {\texttt{test.m}}, kjer sta določena parametrizacija krivulje, po kateri se giba otrok, in njen odvod.

\newpage
\section{\textsc{\large{PRIMERI GIBANJA}}}

Poglejmo si nekaj primerov gibanja. Modra barva krivulje označuje gibanje otroka, rdeča pa igrače.

\subsection{\textsc{Primer 1}}
    
     Gibanje otroka po krivulji s parametrizacijo:
    \begin{center}
    $x(t) = cos(t)$, \\
    $y(t) = sin(t)$. 
    \end{center}
    Če se otrok premika v krogu, igrača pa je na začetku v središču kroga (torej je vrvica dolga enako kot radij - predpostavka), ostane igrača ves čas na istem mestu.
    \\
    \\
    \begin{figure}[!h]
        \centering
        \includegraphics[scale=0.4]{Primer1}
    \end{figure}

    \newpage
   
    \subsection{\textsc{Primer 2}}
     Gibanje otroka po krivulji s parametrizacijo:
    \begin{center}
    $x(t) = t + cos(t)$, \\
    $y(t) = t + sin(t)$. 
    \end{center}      
    \begin{figure}[!h]
        \centering
        \includegraphics[scale=0.4]{Primer2}
    \end{figure}

    \subsection{\textsc{Primer 3}}
    Gibanje otroka po krivulji s parametrizacijo:
    \begin{center}
    $x(t) = t + cos(t)$, \\
    $y(t) = t - sin(t)$. 
    \end{center}      
    \begin{figure}[!h]
        \centering
        \includegraphics[scale=0.4]{Primer3}
    \end{figure}
    
    \newpage
    \subsection{\textsc{Primer 4}}
    Gibanje otroka po krivulji s parametrizacijo:
    \begin{center}
    $x(t) = t + cos(t) + sin(t)$, \\
    $y(t) = t + sin(t) - cos(t)$. 
    \end{center}      
    \begin{figure}[!h]
        \centering
        \includegraphics[scale=0.4]{Primer4}
    \end{figure}
    
    \subsection{\textsc{Primer 5}}
    Gibanje otroka po krivulji s parametrizacijo:
    \begin{center}
    $x(t) = t + cos(t) - sin(t)$, \\
    $y(t) = t + sin(t) + cos(t)$. 
    \end{center}      
    \begin{figure}[!h]
        \centering
        \includegraphics[scale=0.4]{Primer5}
    \end{figure}

\newpage

\section{\textsc{\large{ZAKLJUČEK}}}


\newpage	
\begin{thebibliography}{99}
	\bibitem{a} 
	J Cimprič: Diferencialne enačbe, FMF, skripta, dostopno na https://www.fmf.uni-lj.si/~cimpric/skripta/del6.pdf.

\end{thebibliography}

\end{document}