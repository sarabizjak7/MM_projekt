\documentclass[a4paper,12pt]{article}

\usepackage[slovene]{babel}
\usepackage[utf8]{inputenc}
\usepackage[T1]{fontenc}
\usepackage{lmodern}
\usepackage{amsmath}
\usepackage{amssymb}
\usepackage{amsthm}
\usepackage{amsfonts}
\usepackage{mathtools}
\usepackage{enumitem}



\begin{document}


%%% naslovna stran

\begin{titlepage}
    \centering
    \vfill
    \vfill
    \textbf{\Huge{Poročilo projekta}}
    \vfill
    \textbf{\LARGE{Matematično modeliranje}}
    \vfill\vfill\vfill\vfill\vfill
    \textsc{\Large{Sara Bizjak}}
    \vfill\vfill
    \textsc{\large{Univerza v Ljubljani}}
    
    \textsc{\large{Fakulteta za matematiko in fiziko}}
    
    \textsc{\large{Oddelek za matematiko}}
    \vfill\vfill
        
    \large{Avgust 2019}
    
    \end{titlepage}

\newpage

\tableofcontents

\newpage

\section{\textsc{\large{Predstavitev problema}}}


Rešujemo problem otroka, ki se sprehaja po ravnem igrišču na mivki, za seboj pa vleče na 
vrvico privezano igračo tako, da je vrvica vseskozi napeta. Otrokovo gibanje opišemo s 
parametrično krivuljo. Program izračuna sled gibanja igrače po mivki in izriše animacijo.
\\
Poznamo parametrizacijo krivulje, po kateri se giba otrok. Ker je vrvica med otrokom in 
igračo vedno napeta, vemo, da imata v vsakem trenutku isto hitrost, oz. sta vedno enako 
oddaljena. Igrača se vedno giba v smeri otroka. 
Torej poznamo smer in velikost hitrosti igrače, njene pozicije oz. krivuljo gibanja pa 
dobimo kot rešitev diferencialne enačbe.

\section{\textsc{\large{Reševanje navadnih diferencialnih enačb TODO}}}

Ponovimo osnove pri reševanju diferencialnih enačb. \\
Denimo, da rešujemo začetni problem 

\begin{displaymath}
y'(x) = f(x), \quad y(x_{0}) = y_{0}.
\end{displaymath}
Če je diferencialna enačba višjega reda, jo prevedemo na sistem enačb prvega reda. \\
Za rešitev diferencialne enačbe lahko v Matlabu uporabimo že vgrajeno funkcijo 
ode45:

\begin{displaymath}
[X, Y] = ode45(odefun, [x0, b], y0)
\end{displaymath}

\section{\textsc{\large{Rešitev problema}}}

Z $x(t)$ in $y(y)$ je označena parametrizacija otroka, z $x_i(t)$ in $y_i(t)$ pa igrače. Odvodi so označeni kot $dx(t), \ dy(t), \ dx_i(t)$ in $dy_i(t)$. \\
Poznamo parametrizacijo otroka, torej $x(t)$ in $y(t)$. Vemo, da vektor hitrosti igrače kaže v smeri proti otroku (po vrvici).
Torej velja

\begin{align*}
    dx_i(t) = x(t) - x_i(t) \\
    dy_i(t) = y(t) - y_i(t).
\end{align*}
Vemo tudi, da sta hitrost otroka in hotrost igrače po velikosti enaki, torej velja
\begin{align*}
    \|(dx_i(t), dy_i(t))\| = \|(dx(t), dy(y))\|.
\end{align*}
Ker poznamo smer igrače in velikost hitrosti, lahko ugotovimo njeno parametrizacijo oziroma gibanje.
Vektor smeri normiramo in ga pomnožimo z normo hitrosti otroka, saj se velikosti hitrosti ujemata.
\\
\begin{align}
     \Big(
    \begin{bmatrix} 
        x(t) \\
        y(t)
    \end{bmatrix}
    -
    \begin{bmatrix} 
        x_i(t) \\
        y_i(t) 
    \end{bmatrix}
    \Big)
    \ \cdot \
    \frac{
    \Big \|
    \begin{bmatrix} 
        dx(t) \\
        dy(t)
    \end{bmatrix}
    \Big \|
    }
    {
    \Big \|
    \begin{bmatrix} 
        x(t) \\
        y(t)
    \end{bmatrix}
    -
    \begin{bmatrix} 
        x_i(t) \\
        y_i(t) 
    \end{bmatrix}
    \Big \|
    }
\end{align}
\\
\\
Da ohranjamo razdaljo oz. napeto vrvico med otrokom in igračo, račun (1) pomnožimo še 
s kotom med hitrostjo otroka in daljico, ki povezuje igračo in otroka. 
Velikost kota dobimo s skalarnim produktom, in sicer

\begin{align}
    \varphi = 
    \frac{
    \begin{bmatrix} 
        dx(t), \ %
        dy(t)
    \end{bmatrix}
    \cdot
    \begin{bmatrix} 
        x(t) - x_i(t) \\
        y(t) - y_i(t)
    \end{bmatrix}
    }
    {
    \Big \|
    \begin{bmatrix} 
        dx(t) \\
        dy(t)
    \end{bmatrix}
    \Big \| 
    \cdot
    \Big \|
    \begin{bmatrix} 
        x(t) - x_i(t) \\
        y(t) - y_i(t)
    \end{bmatrix}
    \Big \| 
    }
    .
\end{align}
\\
\\
Če pomnožimo (1) in (2) dobimo sistem \\
\begin{align}
    \begin{bmatrix} 
        dx_i(t) \\
        dy_i(t) 
    \end{bmatrix}
    =
    \frac{
    \begin{bmatrix} 
        x(t) - x_i(t) \\
        y(t) - y_i(t)
    \end{bmatrix}
    \cdot
    \Big(
    \begin{bmatrix} 
        dx(t), \ %
        dy(t)
    \end{bmatrix}
    \cdot
    \begin{bmatrix} 
        x(t) - x_i(t) \\
        y(t) - y_i(t)
    \end{bmatrix}
    \Big)
    }
    {
    \Big \|
    \begin{bmatrix} 
        x(t) - x_i(t) \\
        y(t) - y_i(t)
    \end{bmatrix}
    \Big \| ^ {2}
    }
    .
\end{align}
\\
\\
Sistem (3) sovpada s funkcijo, ki jo ustavimo v Matlabovo vgrajeno funkcijo za reševanje diferencialnih enačb 
\textbf{ode45}.
\\
\\
\begin{align*}
    odefun = @(t,\ P) 
    \
    \frac{
        \begin{bmatrix} 
            x(t) - x_i(t) \\
            y(t) - y_i(t)
        \end{bmatrix}
        \cdot
        \Big(
        \begin{bmatrix} 
            dx(t), \ %
            dy(t)
        \end{bmatrix}
        \cdot
        \begin{bmatrix} 
            x(t) - x_i(t) \\
            y(t) - y_i(t)
        \end{bmatrix}
        \Big)
        }
        {
        \Big \|
        \begin{bmatrix} 
            x(t) - x_i(t) \\
            y(t) - y_i(t)
        \end{bmatrix}
        \Big \| ^ {2}
        },
\end{align*}

\begin{align*}
    kjer \ P = 
        \begin{bmatrix} 
            x_i(t) \\
            y_i(t)
        \end{bmatrix}
    .
    \\
\end{align*}
Pri reševanju je predpostavljeno, da je vrvica dolga toliko, kot je začetna oddaljenost igrače in otroka. 
Igrača svoje gibanje vedno začne v koordinatnem izhodišču, torej v točki (0, 0), kar je tudi začetni pogoj pri reševanju nastavljene diferencialne enačbe.


\section{\textsc{\large{Primeri}}}

Poglejmo si nekaj primerov gibanja. Modra barva krivulje označuje gibanje otroka, rdeča pa igrače.

\begin{itemize}
    \item Gibanje otroka po krivulji s parametrizacijo:
    \begin{center}
    $x(t) = cos(t)$, \\
    $y(t) = sin(t)$. 
    \end{center}
    Če se otrok premika v krogu, igrača pa je na začetku v središču kroga (torej je vrvica dolga enako kot radij - predpostavka), ostane igrača ves čas na istem mestu.
    \begin{figure}[!h]
        \centering
        \includegraphics[scale=0.4]{Primer1}
    \end{figure}

    \newpage
    \item Gibanje otroka po krivulji s parametrizacijo:
    \begin{center}
    $x(t) = t + cos(t)$, \\
    $y(t) = t + sin(t)$. 
    \end{center}      
    \begin{figure}[!h]
        \centering
        \includegraphics[scale=0.4]{Primer2}
    \end{figure}

    \item Gibanje otroka po krivulji s parametrizacijo:
    \begin{center}
    $x(t) = t + cos(t)$, \\
    $y(t) = t - sin(t)$. 
    \end{center}      
    \begin{figure}[!h]
        \centering
        \includegraphics[scale=0.4]{Primer3}
    \end{figure}
    
    \newpage
    \item Gibanje otroka po krivulji s parametrizacijo:
    \begin{center}
    $x(t) = t + cos(t) + sin(t)$, \\
    $y(t) = t + sin(t) - cos(t)$. 
    \end{center}      
    \begin{figure}[!h]
        \centering
        \includegraphics[scale=0.4]{Primer4}
    \end{figure}
    
    \item Gibanje otroka po krivulji s parametrizacijo:
    \begin{center}
    $x(t) = t + cos(t) - sin(t)$, \\
    $y(t) = t + sin(t) + cos(t)$. 
    \end{center}      
    \begin{figure}[!h]
        \centering
        \includegraphics[scale=0.4]{Primer5}
    \end{figure}

\end{itemize}



\end{document}